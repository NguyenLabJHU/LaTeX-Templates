\documentclass[12pt,letterpaper]{report}    % report document class w/ 12 pt font

%%%%%%%%%%%%%%%%%%%%%%%%%%%%%%%%%%%%%%%%%%%%%%%%%%%%%%%%%%%%%%%%%%%%%%%%%%%%%%%%
%% if possible, make your formatting changes here through the variables 
%%%%%%%%%%%%%%%%%%%%%% LIST OF VARIABLES FOR FORMATTING %%%%%%%%%%%%%%%%%%%%%%%%

\def\GlobalMargin{1.0in}                    % margin on all sides
\def\PrintingOffset{0in}                    % additional left margin for the printed copy
\def\MainTextSpacing{\singlespacing}        % double-spaced main text


%%%% FONT TYPE and SIZE
\def\FontPackage{lmodern}                   % Latin Modern font

%%%% ADDITIONAL PATHS and FILES
\def\FigurePath{figures}                    % subdirectory for the figure files
\def\BibFileName{template.bib}                % name of BibLaTeX file 


%%%% SECTION LEVELS and TOC APPEARANCE
\def\NoSectionLevel{3}                      % 3 levels for sections ... to subsubsection
\def\NoTocLevel{2}                          % no of levels showed in the table of contents
%% 2 levels mean: section and subsection.. decrease if you want to show less in TOC


%%%% FONT SIZE and TYPESET for DIFFERENT HEADINGS
%% check here for details: https://en.wikibooks.org/wiki/LaTeX/Fonts
\def\TitleFont{\Large\bfseries\MakeUppercase}   % font format for thesis title
\def\ChapterFont{\Large\bfseries\singlespacing} % font format for chapter label and title
\def\SectionFont{\large\bfseries}           % section heading font format
\def\SubsectionFont{\normalsize\bfseries}   % subsection heading font format
\def\SubsubsectionFont{\normalsize\itshape} % subsubsection heading font format
\def\CaptionFontSize{small}                 % caption font size
\def\CaptionFontType{bf}                    % boldface label for captions
\def\CaptionSeparator{colon}                % separates caption heading from text. can use 'period' as well

%%%% spacing between different common environments
\def\ParagraphSpacing{\baselineskip}        % spacing between paragraph
\def\ParagraphIndent{0 pt}                  % indentation at the beginning of the paragraph
\def\CaptionSpacing{0 pt}                   % spacing between the figure and the caption
\def\FootnoteSpacing{\baselineskip}         % spacing between footnotes



%%%% TOC SPACING of different section levels (chapter to subsubsection)
\def\TOCTextSpacing{\singlespacing}         % single spacing for TOC texts
\def\ChapTOCSpacing{\baselineskip}          % spacing between chapters
\def\SecTOCSpacing{0.5\baselineskip}        % spacing between sections
\def\SubsecTOCSpacing{0.3\baselineskip}     % ... between subsections
\def\SubsubsecTOCSpacing{0.3\baselineskip}  % ... between subsubsections
\def\TocIndent{0 pt}                        % indentation in the list of figs and tables
\def\LOTItemSpacing{\baselineskip}          % spacing between LOT/LOF items


%%%% CHAPTER QUOTE (EPIGRAPH PACKAGE)
\def\ChapQuoteFontSize{\small}              % font size of chapter quotes
\def\ChapQuoteLocation{flushright}          % location of chapter quote
\def\ChapQuoteTextShape{\itshape}           % font shape for quotes
\def\ChapQuoteAuthorTextShape{\scshape}     % font shape for quote author
\def\MaxQuoteWidth{0.65\textwidth}          % width epigraph-based quotes in chapter

%%%% BIBLIOGRAPHY ITEMS
\def\BibTextSpacing{\singlespacing}         % single-spaced bibliography
\def\BibItemSpacing{0.5\baselineskip}       % spacing between bibliographic items in reference


%%%% GLOBAL SPACING for TABLES
\def\GlobalTableSpacing{1.5}                % global spacing parameter for table
%% if this seems too widespread for you, try changing it locally using
%% \begin{group} ... \renewcommand{\arraystretch} ... \end{group} commands



%%%% CHAPTER HEADER SETTINGS
\def\HeaderHeight{30 pt}                    % height of the chapter header
\def\HeaderSpace{12 pt}                     % space between header and the following text


%%%% ADHOC HEIGHT ADJUSTMENT VARIABLES for consistent formatting
%% following numbers are found by trial and error to compensate 
%% for default spacing around default LaTeX environments
\def\TitleTopSpacing{-\HeaderHeight-\HeaderSpace}   % top height adjustment for thesis title
\def\BeforeTOCTitleSpacing{-31 pt}          % space before TOC title
\def\AfterTOCTitleSpacing{34 pt}            % space after TOC title
\def\AfterLOTTitleSpacing{34 pt}            % space after LOT/ LOF title

\def\NumChapterTopMargin{-54 pt}            % space before numbered chapter label
\def\UnNumChapterTopMargin{-75 pt}          % space before unnumbered chapter label
\def\ChapLabelToTitle{-27 pt}               % space between chapter label to title 
\def\ChapTitleToText{24 pt}                 % space between the chapter title and the following text
\def\SpaceBeforeQuote{-10 pt}               % white space before quote

%%%%%%%%%%%%%%%%%%%% END LIST OF VARIABLES FOR FORMATTING %%%%%%%%%%%%%%%%%%%%%%


%%%%%%%%%%%%%%%%%%%%%%%%%%%%%%%%%%%%%%%%%%%%%%%%%%%%%%%%%%%%%%%%%%%%%%%%%%%%%%%%
%% add packages as needed but sometimes the order of the packages matters.
%% you may have to change the options of the biblatex package for the bibliography.
%%%%%%%%%%%%%%%%%%%%%%%%%%%%%%% LaTeX PACKAGES %%%%%%%%%%%%%%%%%%%%%%%%%%%%%%%%%

%%%% SOME PRE-REQUISITE PACKAGES
\usepackage[utf8]{inputenc}                 % for encoding input character (required)
\usepackage[american]{babel}                % for different language typography
\usepackage[T1]{fontenc}                    % for font encoding


%%%% DEFAULT FONT PACKAGE (you can add something else ignoring \FontPackage variable)
\usepackage{\FontPackage}
% \usepackage[sc]{mathpazo}                 % Palatino font for example


%%%% COMMON MATH PACKAGES
\usepackage{amsfonts,amssymb,amsmath,amsthm,autobreak,
cancel,dsfont,mathtools,mathbbol,mathrsfs,siunitx,upgreek}


%%%% BIBLIOGRAPHIC PACKAGE (change the style or other options if you need to)
%% Nature style bibliography
\usepackage[backend=biber, defernumbers=true, style=nature, maxnames=99,
     date=year, isbn=false, url=false, doi=true]{biblatex}

%% APA style
% \usepackage[backend=biber, defernumbers=true, style=apa, 
%   isbn=false, url=false, doi=true]{biblatex}

%% IEEE style
% \usepackage[backend=biber,style=ieee,defernumbers=true,maxnames=99, 
%   date=year, isbn=false, url=false, doi=true]{biblatex}


%%%% TABLE-RELATED PACKAGES
\usepackage{booktabs,longtable,dcolumn,makecell,
multicol,multirow,tabularx,xltabular,rotating}



%%%% package for micro-typography (you can define more settings)
%% see details here: https://www.khirevich.com/latex/microtype/
\usepackage[activate={true,nocompatibility}]{microtype}



%%%% OTHER PACKAGES AND OPTIONS
\usepackage[pagewise,mathlines]{lineno}     % line numbers for drafting
\usepackage[ruled]{algorithm2e}             % to manage algorithm environment
\usepackage[titletoc]{appendix}             % to manage appendix chapters
\usepackage{blindtext}                      % to generate random filler texts
\usepackage{calc}                           % to set arithmetic arguments for spacing
\usepackage{caption}                        % to manage captions
\usepackage{comment}                        % to comment a large amount of text as env
\usepackage{epigraph,varwidth}              % for managing quotes
\usepackage{enumitem}                       % to manage list environment
\usepackage{float}                          % to manage floating environment
% footnote environment management
\usepackage[bottom,multiple,hang,flushmargin]{footmisc}      
\usepackage{graphicx,wrapfig}               % to manage images
\usepackage{geometry}                       % to manage margins and page format
\usepackage{glossaries}                     % to add glossaries
\usepackage{fancyhdr}                       % for header/ footer settings
\usepackage[dvipsnames]{xcolor}             % color-related package
\usepackage[pdfa]{hyperref}                 % for hyperlink management 
\usepackage[all]{hypcap}                    % for captions on the side of figures
\usepackage{ifthen}                         % if-then statement in LaTeX code
\usepackage{lscape}                         % landscape mode
\usepackage{listings,minted}                % to include codes 
\usepackage{csquotes}                       % yet another package to manage quote
\usepackage{setspace}                       % sets space between lines
\usepackage{seqsplit}                       % splits long character sequence
\usepackage[rightcaption]{sidecap}          % for sideway captions
\usepackage{tocloft}                        % to manage table of contents, etc.
\usepackage{textcomp}                       % text companion fonts in TS1
\usepackage[absolute]{textpos}              % position text at certain location
\usepackage{titlesec}                       % managing different title environments
\usepackage[final]{pdfpages}                % to insert pdf pages
\usepackage{parskip}                        % default spacing around environments
\usepackage{tikz}                           % drawing related package
\usepackage{subcaption}                     % individual panel and caption

%%%% if you try to customize spacing around different environments consider:
% \usepackage[unit=in,type=upperleftT,color=red,showframe]{fgruler}

%% add more packages and/or change options of the packages as needed

%%%%%%%%%%%%%%%%%%%%%%%%%%%%%% END LaTeX PACKAGES %%%%%%%%%%%%%%%%%%%%%%%%%%%%%



%%%%%%%%%%%%%%%%%%%%%%%%%%%%%%%%%%%%%%%%%%%%%%%%%%%%%%%%%%%%%%%%%%%%%%%%%%%%%%%
%% specifying direct package options related to document formatting
%%%%%%%%%%%%%%%%%%%%%%%%%%%%%% PACKAGE OPTIONS %%%%%%%%%%%%%%%%%%%%%%%%%%%%%%%%

%%%% GRAPHICX package
\graphicspath{{\FigurePath/}}


%%%% GEOMETRY PACKAGE: margin settings
\geometry{letterpaper, margin=\GlobalMargin, bindingoffset=\PrintingOffset, 
    nomarginpar, includehead, headheight=\HeaderHeight, 
    headsep=\HeaderSpace, includefoot, heightrounded}
%% you can add showframe option to see how the layout looks like


%%%% HYPERREF PACKAGE
\hypersetup{linktocpage, unicode, linktoc=all, colorlinks=true, 
    citecolor=blue, filecolor=blue, linkcolor=blue, urlcolor=blue}
\urlstyle{rm}           % removes default \texttt style for hyperlinks


%%%% CAPTION PACKAGE
\captionsetup{belowskip=\CaptionSpacing, font=\CaptionFontSize, 
    labelfont=\CaptionFontType, labelsep=\CaptionSeparator, hypcap=true} 



%%%% BIBLATEX: bibliography package settings
\addbibresource{\BibFileName}           % name of the bib file 
\DeclareFieldFormat{titlecase}{\MakeSentenceCase*{#1}}
\AtBeginBibliography{\urlstyle{rm}}     % roman font family for URL (DOI)
\AtBeginBibliography{\vspace*{8pt}}     % add space for single-spaced bib text


%% separate category for papers to be not cited in the bibliography
\DeclareBibliographyCategory{mypapers}             
\newcommand{\mybibexclude}[1]{\addtocategory{mypapers}{#1}}


%%%% SOME SAMPLE TikZ LIBRARY OPTIONS
\usetikzlibrary{positioning,shapes,arrows}

%%%%%%%%%%%%%%%%%%%%%%%%%%%% END PACKAGE OPTIONS %%%%%%%%%%%%%%%%%%%%%%%%%%%%%%%


%%%%%%%%%%%%%%%%%%%%%%%%%%%%%%%%%%%%%%%%%%%%%%%%%%%%%%%%%%%%%%%%%%%%%%%%%%%%%%%
%% further tweaking variables for the current consistent document formatting
%%%%%%%%%%%%%%%%%%%%%%%%%%%%% DOCUMENT FORMATTING %%%%%%%%%%%%%%%%%%%%%%%%%%%%%%

\setcounter{tocdepth}{\NoTocLevel}                  % list depth in ToC
\setcounter{secnumdepth}{\NoSectionLevel}           % section to ... subsubsection ...


%%%% font size and spacing around the titles of ToC/ LoT/ LoF
\renewcommand{\cfttoctitlefont}{\ChapterFont}
\renewcommand{\cftlottitlefont}{\ChapterFont}
\renewcommand{\cftloftitlefont}{\ChapterFont}

\setlength{\cftbeforetoctitleskip}{\BeforeTOCTitleSpacing}
\setlength{\cftbeforelottitleskip}{\BeforeTOCTitleSpacing}
\setlength{\cftbeforeloftitleskip}{\BeforeTOCTitleSpacing}
\setlength{\cftaftertoctitleskip}{\AfterTOCTitleSpacing}
\setlength{\cftafterlottitleskip}{\AfterLOTTitleSpacing}
\setlength{\cftafterloftitleskip}{\AfterLOTTitleSpacing}


%% tweak to TOC to add 'chapter' to the chapter name instead of a number only
%% set the width of the box based on the longest label name
\renewcommand{\cftchappresnum}{\chaptername\space}
\renewcommand{\cftchapleader}{\cftdotfill{\cftdotsep}}  % dots for chapters too
\setlength{\cftchapnumwidth}{\widthof{\textbf{Appendix~XXX~}}}


\setlength{\cftbeforechapskip}{\ChapTOCSpacing}
\setlength{\cftbeforesecskip}{\SecTOCSpacing}
\setlength{\cftbeforesubsecskip}{\SubsecTOCSpacing}
\setlength{\cftbeforesubsubsecskip}{\SubsubsecTOCSpacing}


%% tweak to LOT and LOF to add 'Table'/ 'Figure' to the table/ figure caption listing
%% to change the distance to the start of the table/ figure title
\setlength{\cfttabindent}{\TocIndent}               % indentation from tables in LoT
\renewcommand{\cfttabpresnum}{\bfseries Table }
\setlength{\cfttabnumwidth}{\widthof{\textbf{Table~999.999~}}}
\setlength{\cftbeforetabskip}{\LOTItemSpacing}      % spacing between each item


\setlength{\cftfigindent}{\TocIndent}               % indentation from figures in LoF
\renewcommand{\cftfigpresnum}{\bfseries Figure }
\setlength{\cftfignumwidth}{\widthof{\textbf{Figure~999.999~}}}
\setlength{\cftbeforefigskip}{\LOTItemSpacing}      % spacing between each item


%%%% TITLESEC: settings for chapter label and title
\titleformat{\chapter}[display]{\ChapterFont}
    {\chaptertitlename\ \thechapter}{\ChapLabelToTitle}{\ChapterFont}

%
\titlespacing*{\chapter}{0pt}{\NumChapterTopMargin}
    {\ChapTitleToText} 
%
\titlespacing*{name=\chapter,numberless}{0pt}
    {\UnNumChapterTopMargin}{\ChapTitleToText}


%%%% TITLESEC: settings for sections, subsection, ... heading format
\titleformat*{\section}{\SectionFont}
\titleformat*{\subsection}{\SubsectionFont}
\titleformat*{\subsubsection}{\SubsubsectionFont}
%% if you had more levels then add settings for paragraph and subparagraph


%% to customize space around section headings, use the following command:
% \titlespacing*{environment-name}{space-left}{space-before}{space-after}


%%%% PARSKIP: for paragraph (and not title) spacing, roughly speaking
\renewcommand{\arraystretch}{\GlobalTableSpacing}   % spacing inside table
\setlength{\parskip}{\ParagraphSpacing}             % paragraph skip
\setlength{\parindent}{\ParagraphIndent}            % paragraph indentation
\setlength{\bibitemsep}{\BibItemSpacing}            % bib item separation 
\setlength{\footnotesep}{\FootnoteSpacing}          % separation between footnote


%%%% settings for math environment
\allowdisplaybreaks[1]                  % page break for long equations
\numberwithin{equation}{chapter}        % eqn no with chapter label
\setcounter{MaxMatrixCols}{20}          % no of maximum columns in matrix

%%%%%%%%%%%%%%%%%%%%%%%%%%% END DOCUMENT FORMATTING %%%%%%%%%%%%%%%%%%%%%%%%%%%



%%%%%%%%%%%%%%%%%%%%%%%%%%%%%% OTHER MACROS %%%%%%%%%%%%%%%%%%%%%%%%%%%%%%%%%%%

%%%% UNNUMBERED CHAPTERS, SECTION, and SUBSECTION COMMAND for ADDING to TOC
%% removes the 'Chapter #' title while keeping it listed in the TOC
\newcommand\chap[1]{%
    \chapter*{#1}%
    \markboth{#1}{}
    \addcontentsline{toc}{chapter}{#1}}
  
%% removes the 'Section #' title while keeping it listed in the TOC
\newcommand\sect[1]{%
    \phantomsection
    \section*{#1}%
    \addcontentsline{toc}{section}{#1}}
  
%% removes the 'Subsection #' title while keeping it listed in the TOC
\newcommand\subsect[1]{%
    \phantomsection
    \subsection*{#1}%
    \addcontentsline{toc}{subsection}{#1}}

%% removes the 'Subsubsection #' title while keeping it listed in the TOC
\newcommand\subsubsect[1]{%
    \phantomsection
    \subsubsection*{#1}%
    \addcontentsline{toc}{subsubsection}{#1}}
    

%%%% TOCLOFT: modified macros/ commands for printing ToC, LoF, LoT
\newcommand{\mytableofcontents}{
    \clearpage
    \renewcommand{\contentsname}{Table of Contents}
    \tableofcontents
    \clearpage
}
%
\newcommand{\mylistoffigures}{
    \clearpage \phantomsection
    \addcontentsline{toc}{chapter}{List of Figures}
    \listoffigures
    \clearpage
}
%
\newcommand{\mylistoftables}{
    \clearpage \phantomsection
    \addcontentsline{toc}{chapter}{List of Tables}
    \listoftables
    \clearpage
}

%%%% IN-TEXT MACROS for notes
\newcommand{\COMMENT}{\textcolor{red}}
\newcommand{\ADDCITATION}{\COMMENT{(ADD CITATION)}}

%%%%%%%%%%%%%%%%%%%%%%%%%%%% END OTHER MACROS %%%%%%%%%%%%%%%%%%%%%%%%%%%%%%%%%


%%%%%%%%%%%%%%%%%%%%%%%%%%%%%%%%%%%%%%%%%%%%%%%%%%%%%%%%%%%%%%%%%%%%%%%%%%%%%%%
%% only if you plan on using chapter quotes, you may need epigraph settings
%%%%%%%%%%%%%%%%%%%%%%%%%%%%% EPIGRAPH SETTINGS %%%%%%%%%%%%%%%%%%%%%%%%%%%%%%%

\renewcommand{\epigraphflush}{\ChapQuoteLocation}       % chapter epigraph on right 
\renewcommand{\epigraphsize}{\ChapQuoteFontSize}        % font size for chapter epigraph
\setlength{\epigraphwidth}{\MaxQuoteWidth}              % max width of chapter epigraph
\renewcommand{\textflush}{\ChapQuoteLocation}
\renewcommand{\sourceflush}{\ChapQuoteLocation}
\newcommand{\epitextfont}{\ChapQuoteTextShape}          % quote font shape
\newcommand{\episourcefont}{\ChapQuoteAuthorTextShape}  % quote author name shape

%% following settings put variable width underline between quote and author
\makeatletter
\setlength{\beforeepigraphskip}{\SpaceBeforeQuote}
\newsavebox{\epi@textbox}
\newsavebox{\epi@sourcebox}
\newlength\epi@finalwidth
\renewcommand{\epigraph}[2]{%
    \vspace{\beforeepigraphskip}
    {\epigraphsize\begin{\epigraphflush}
    \epi@finalwidth=\z@
    \sbox\epi@textbox{%
        \varwidth{\epigraphwidth}
        \begin{\textflush}\epitextfont#1\end{\textflush}
        \endvarwidth
   }%
    \epi@finalwidth=\wd\epi@textbox
    \sbox\epi@sourcebox{%
        \varwidth{\epigraphwidth}
        \begin{\sourceflush}\episourcefont#2\end{\sourceflush}%
        \endvarwidth
   }%
    \ifdim\wd\epi@sourcebox>\epi@finalwidth 
        \epi@finalwidth=\wd\epi@sourcebox
    \fi
   \leavevmode\vbox{
        \hb@xt@\epi@finalwidth{\hfil\box\epi@textbox}
        \vskip 1ex         % gap between quote and rule
        \hrule height \epigraphrule
        \vskip 1ex         % gap between rule and author
        \hb@xt@\epi@finalwidth{\hfil\box\epi@sourcebox}
   }%
   \end{\epigraphflush}
   \vspace{\afterepigraphskip}}}
\makeatother

%%%%%%%%%%%%%%%%%%%%%%%%%%% END EPIGRAPH SETTINGS %%%%%%%%%%%%%%%%%%%%%%%%%%%%%



%%%%%%%%%%%%%%%%%%%%%%%%%%%%%%%%%%%%%%%%%%%%%%%%%%%%%%%%%%%%%%%%%%%%%%%%%%%%%%%
%% if you plan on adding algorithms and codes to your thesis, these settings
%% may be helpful. you can tweak them based on your needs and preferences.
%%%%%%%%%%%%%%%%%%%%%% ALGORITHM AND LISTING SETTINGS %%%%%%%%%%%%%%%%%%%%%%%%%

%% settings for algorithm2e package
\renewcommand{\algorithmcfname}{Procedure}
\SetKwFor{While}{while}{}{end while}%
\SetArgSty{textnormal}
\newcommand\mycommfont[1]{\footnotesize\ttfamily\textcolor{blue}{#1}}
\SetCommentSty{mycommfont}


%% listing package definition (to add code in the document)
\lstdefinestyle{terminal}{columns=fullflexible,
keepspaces=true,
breaklines=true,
basicstyle={\footnotesize\fontfamily{fvm}\fontseries{m}\selectfont},
keywordstyle={\footnotesize\fontfamily{fvm}\fontseries{b}\selectfont},
commentstyle={\color{comments}\small\fontfamily{fvm}\itshape\selectfont},
frame=single,
xleftmargin=0in,
backgroundcolor=\color{lightgray!50},
belowcaptionskip=10pt,
aboveskip=0.5cm}
\lstset{style=terminal,float=h,language=bash}

%%%%%%%%%%%%%%%%%%%% END ALGORITHM AND LISTING SETTINGS %%%%%%%%%%%%%%%%%%%%%%%



%%%%%%%%%%%%%%%%%%%%%%%%%%%%%%%%%%%%%%%%%%%%%%%%%%%%%%%%%%%%%%%%%%%%%%%%%%%%%%%
%% add all your custom math settings and macros in the following section.
%% this is where LaTeX supremacy becomes a thing. you can customize a lot.
%%%%%%%%%%%%%%%%%%%%%%% MATH SETTINGS AND MACROS %%%%%%%%%%%%%%%%%%%%%%%%%%%%%%

\newcommand{\dC}{$^{\circ}$C}           % degree celcius symbol
\newcommand{\vect}[1]{\mathbf{#1}}      % boldface for vectors and tensors
\DeclareMathOperator{\T}{{\top}}        % transpose of a matrix/ tensor
\DeclareMathOperator{\tr}{tr}           % trace of a matrix
\DeclareMathOperator{\divg}{div}        % divergence of vector and tensor
\DeclareMathOperator{\grad}{grad}       % gradient of vector and tensor
\DeclareMathOperator{\curl}{curl}       % curl of vector and tensor

% theorem-style remark environment
\theoremstyle{definition}
\newtheorem{remark}{Remark}

%% these are just some examples; add more macros for your custom symbols

%%%%%%%%%%%%%%%%%%%%% END MATH SETTINGS AND MACROS %%%%%%%%%%%%%%%%%%%%%%%%%%%%
%%%%%%%%%%%%%%%%%%%%%%%%%%%%%%%%%%%%%%%%%%%%%%%%%%%%%%%%%%%%%%%%%%%%%%%%%%%%%%%